\documentclass{article}
\oddsidemargin=1cm
\evensidemargin=1cm
\evensidemargin=1cm
\topmargin=-0.5cm
\headheight=0cm
\headsep=1cm
\textheight=23cm
\parindent=1.5cm
\begin{document}
\title{Interview}
\author{Hitesh Agarwal }
\date{12,July 2019}
\maketitle



\section*{1. Choice of Interviewee}

\noindent I have chosen Kunal Singla as my interviewee. He is a friend of mine who is pursuing MSc(Mathematics Hons) from IIT Delhi.
\noindent I chose him because of his knowledge and interest in mathematics and his qualifications speak for itself. Moreover, he is a tech savvy person who is always interested in new applications.

\section*{2. Interview Questions and Answers}
\noindent \textbf {Question 1) Could you state your name and your qualifications please ?\\*}
\noindent Answer 1) My name is Kunal Singla. I have done BSc(Mathematics Hons) from Punjab University, Chandigarh, India and I am currently pursuing my MSc(Mathematics Hons) from IIT,Delhi.
\begin{flushleft}
\hrulefill
\end{flushleft}

\noindent \textbf{Question 2) Are you aware about the reason and goal of this interview ?\\*}
\noindent Answer 2) Yes. I believe you are designing a calculator that computes the value of Euler numbers and the purpose of this interview is to gather information about the functionalities and general design for such a calculator from a users perspective .

\begin{flushleft}
\hrulefill
\end{flushleft}
\noindent \textbf{Question 3) Should the layout of the calculator be the same as the generic type or would you like some modifications done to it?}\\*
\noindent Answer 3) Yes I would like the layout to be same as I am very used to using a generic calculator and it would be best if no major changes were done to it. 
\begin{flushleft}
\hrulefill
\end{flushleft}

\noindent \textbf {Question 4) Apart from the basic mathematical operations, what other operations would you like to have that would complement the     irrational number? \\*}
\noindent Answer 4)It would be great to have a few functions of a scientific calculator, but none in particular. 
\begin{flushleft}
\hrulefill
\end{flushleft}
\noindent \textbf {Question 5) In what specific areas of your studies would you use this calculator? }
\noindent Answer 5) Euler's number is used in various ares of mathematics such as calculus, differential equations, discrete mathematics, trigonometry, complex analysis, statistics.
\begin{flushleft}
\hrulefill
\end{flushleft}
\noindent \textbf {Question 6) Do you want the calculator to solve any trigonometric functions ?\\*}
\noindent Answer 6) No.
\begin{flushleft}
\hrulefill
\end{flushleft}

\noindent \textbf {Question 7) What operations would like to be done automatically by just pushing a key?\\*}
\noindent Answer 7) Euler's number is very useful in calculating compound interest. So it might be useful to have a button to calculate that.
\begin{flushleft}
\hrulefill
\end{flushleft}
\noindent \textbf {Question 8) Can you think of any other irrational number that can be combined with the Euler's number for some common problems?\\*}
\noindent Answer 8) Yes, as a matter of fact Euler's number is an integral part of a calculation involving Guassian Integral, so it would be good if the calculator could also compute the Guassian Integral.
\begin{flushleft}
\hrulefill
\end{flushleft}

 


 
\end{document}


\end{document}
