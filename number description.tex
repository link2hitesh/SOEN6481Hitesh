\documentclass{article}

\usepackage[latin1]{inputenc}
\usepackage{color}
\usepackage{listings}
\usepackage[english]{babel}
\usepackage{graphicx}
\usepackage{indentfirst}

%\usepackage{fontspec}
%\usepackage[utf8]{inputenc}
%\setmainfont{Futura}[ItalicFont={Futura Italic}]

\definecolor{colKeys}{rgb}{0,0,1}
\definecolor{colIdentifier}{rgb}{0,0,0}
\definecolor{colComments}{rgb}{0,0.5,1}
\definecolor{colString}{rgb}{0.6,0.1,0.1}

\lstset{%configuration de listings
float=hbp,
basicstyle=\ttfamily\small, 
identifierstyle=\color{colIdentifier}, 
keywordstyle=\color{colKeys}, 
stringstyle=\color{colString}, 
commentstyle=\color{colComments}, 
columns=flexible, 
tabsize=2, 
%frame=trBL, 
frameround=tttt, 
extendedchars=true, 
showspaces=false, 
showstringspaces=false, 
numbers=left, 
numberstyle=\tiny, 
breaklines=true, 
breakautoindent=true, 
captionpos=b,
xrightmargin= 1cm, 
xleftmargin= 1cm
} 
\lstset{language=c++}
\lstset{commentstyle=\textit}


\parskip 5pt plus 2pt minus 2pt
\textwidth=14cm
\oddsidemargin=1cm
\evensidemargin=1cm
\topmargin=-0.5cm
\headheight=0cm
\headsep=1cm
\textheight=23cm
\parindent=1.5cm

\begin{document}
\thispagestyle{empty}
\begin{center}
Introduction to Euler$'$s number and its unique characteristics
\end{center}
\bigskip


\section*{1. INTRODUCTION}

\noindent  Euler$'$s number e is a mathematical constant which is the base of the natural logarithm. It's a unique number whose natural logarithm is equal to one. It's value is approximately 2.71828 and is the limit of (1 + 1/n)n as n approaches infinity. It can also be calculated as the sum of the infinite series:
$${\displaystyle e=\sum \limits _{n=0}^{\infty }{\frac {1}{n!}}={\frac {1}{1}}+{\frac {1}{1}}+{\frac {1}{1\cdot 2}}+{\frac {1}{1\cdot 2\cdot 3}}+\cdots }\displaystyl}$$

\noindent Euler$'$s number is named after the Swiss mathematician Leonhard Euler, number e is also known as Napier's constant

\section*{2. UNIQUE CHARACTERISTICS}
\noindent 1) e can be defined as a unique positive number $'$a$'$ such that the graph of the function y=ax has unit slope at x = 0.[3] The function f(x) = ex is called the (natural) exponential function, and is the unique exponential function equal to its own derivative.\\*\newline
\noindent 2) There is the remarkable property that if the function $e^ x$ (known as the exponential function and also denoted as "$\exp (x) $") is differentiated with respect to $x$, then the result is the same function $e^ x$

\section*{3. APPLICATIONS}
\noindent 1) Compound Interest: The usage of this constant in compound interest was discovered by James Bernoulli in 1683.Bernoulli noticed that an account that would start with 1\$ with interest rate R will yield [eRt] dollars after t years with continuous compounding.\\*\newline
\noindent 2) Derangements: Discovered by  Pierre Raymond de Montmort and Bernoulli , e could be applied in  derangements. A derangement is a permutation of the elements of a set, such that no element appears in its original position.\\*\newline
\noindent 3) Standard normal distribution: The normal distribution with zero mean and unit standard deviation is known as the standard normal distribution, given by the probability density function

{\displaystyle \phi (x)={\frac {1}{\sqrt {2\pi }}}e^{-{\frac {1}{2}}x^{2}}.} {\displaystyle \phi (x)={\frac {1}{\sqrt {2\pi }}}e^{-{\frac {1}{2}}x^{2}}.}
\end{document}





